
% Default to the notebook output style

    


% Inherit from the specified cell style.




    
\documentclass[11pt]{article}

    
    
    \usepackage[T1]{fontenc}
    % Nicer default font (+ math font) than Computer Modern for most use cases
    \usepackage{mathpazo}

    % Basic figure setup, for now with no caption control since it's done
    % automatically by Pandoc (which extracts ![](path) syntax from Markdown).
    \usepackage{graphicx}
    % We will generate all images so they have a width \maxwidth. This means
    % that they will get their normal width if they fit onto the page, but
    % are scaled down if they would overflow the margins.
    \makeatletter
    \def\maxwidth{\ifdim\Gin@nat@width>\linewidth\linewidth
    \else\Gin@nat@width\fi}
    \makeatother
    \let\Oldincludegraphics\includegraphics
    % Set max figure width to be 80% of text width, for now hardcoded.
    \renewcommand{\includegraphics}[1]{\Oldincludegraphics[width=.8\maxwidth]{#1}}
    % Ensure that by default, figures have no caption (until we provide a
    % proper Figure object with a Caption API and a way to capture that
    % in the conversion process - todo).
    \usepackage{caption}
    \DeclareCaptionLabelFormat{nolabel}{}
    \captionsetup{labelformat=nolabel}

    \usepackage{adjustbox} % Used to constrain images to a maximum size 
    \usepackage{xcolor} % Allow colors to be defined
    \usepackage{enumerate} % Needed for markdown enumerations to work
    \usepackage{geometry} % Used to adjust the document margins
    \usepackage{amsmath} % Equations
    \usepackage{amssymb} % Equations
    \usepackage{textcomp} % defines textquotesingle
    % Hack from http://tex.stackexchange.com/a/47451/13684:
    \AtBeginDocument{%
        \def\PYZsq{\textquotesingle}% Upright quotes in Pygmentized code
    }
    \usepackage{upquote} % Upright quotes for verbatim code
    \usepackage{eurosym} % defines \euro
    \usepackage[mathletters]{ucs} % Extended unicode (utf-8) support
    \usepackage[utf8x]{inputenc} % Allow utf-8 characters in the tex document
    \usepackage{fancyvrb} % verbatim replacement that allows latex
    \usepackage{grffile} % extends the file name processing of package graphics 
                         % to support a larger range 
    % The hyperref package gives us a pdf with properly built
    % internal navigation ('pdf bookmarks' for the table of contents,
    % internal cross-reference links, web links for URLs, etc.)
    \usepackage{hyperref}
    \usepackage{longtable} % longtable support required by pandoc >1.10
    \usepackage{booktabs}  % table support for pandoc > 1.12.2
    \usepackage[inline]{enumitem} % IRkernel/repr support (it uses the enumerate* environment)
    \usepackage[normalem]{ulem} % ulem is needed to support strikethroughs (\sout)
                                % normalem makes italics be italics, not underlines
    

    
    
    % Colors for the hyperref package
    \definecolor{urlcolor}{rgb}{0,.145,.698}
    \definecolor{linkcolor}{rgb}{.71,0.21,0.01}
    \definecolor{citecolor}{rgb}{.12,.54,.11}

    % ANSI colors
    \definecolor{ansi-black}{HTML}{3E424D}
    \definecolor{ansi-black-intense}{HTML}{282C36}
    \definecolor{ansi-red}{HTML}{E75C58}
    \definecolor{ansi-red-intense}{HTML}{B22B31}
    \definecolor{ansi-green}{HTML}{00A250}
    \definecolor{ansi-green-intense}{HTML}{007427}
    \definecolor{ansi-yellow}{HTML}{DDB62B}
    \definecolor{ansi-yellow-intense}{HTML}{B27D12}
    \definecolor{ansi-blue}{HTML}{208FFB}
    \definecolor{ansi-blue-intense}{HTML}{0065CA}
    \definecolor{ansi-magenta}{HTML}{D160C4}
    \definecolor{ansi-magenta-intense}{HTML}{A03196}
    \definecolor{ansi-cyan}{HTML}{60C6C8}
    \definecolor{ansi-cyan-intense}{HTML}{258F8F}
    \definecolor{ansi-white}{HTML}{C5C1B4}
    \definecolor{ansi-white-intense}{HTML}{A1A6B2}

    % commands and environments needed by pandoc snippets
    % extracted from the output of `pandoc -s`
    \providecommand{\tightlist}{%
      \setlength{\itemsep}{0pt}\setlength{\parskip}{0pt}}
    \DefineVerbatimEnvironment{Highlighting}{Verbatim}{commandchars=\\\{\}}
    % Add ',fontsize=\small' for more characters per line
    \newenvironment{Shaded}{}{}
    \newcommand{\KeywordTok}[1]{\textcolor[rgb]{0.00,0.44,0.13}{\textbf{{#1}}}}
    \newcommand{\DataTypeTok}[1]{\textcolor[rgb]{0.56,0.13,0.00}{{#1}}}
    \newcommand{\DecValTok}[1]{\textcolor[rgb]{0.25,0.63,0.44}{{#1}}}
    \newcommand{\BaseNTok}[1]{\textcolor[rgb]{0.25,0.63,0.44}{{#1}}}
    \newcommand{\FloatTok}[1]{\textcolor[rgb]{0.25,0.63,0.44}{{#1}}}
    \newcommand{\CharTok}[1]{\textcolor[rgb]{0.25,0.44,0.63}{{#1}}}
    \newcommand{\StringTok}[1]{\textcolor[rgb]{0.25,0.44,0.63}{{#1}}}
    \newcommand{\CommentTok}[1]{\textcolor[rgb]{0.38,0.63,0.69}{\textit{{#1}}}}
    \newcommand{\OtherTok}[1]{\textcolor[rgb]{0.00,0.44,0.13}{{#1}}}
    \newcommand{\AlertTok}[1]{\textcolor[rgb]{1.00,0.00,0.00}{\textbf{{#1}}}}
    \newcommand{\FunctionTok}[1]{\textcolor[rgb]{0.02,0.16,0.49}{{#1}}}
    \newcommand{\RegionMarkerTok}[1]{{#1}}
    \newcommand{\ErrorTok}[1]{\textcolor[rgb]{1.00,0.00,0.00}{\textbf{{#1}}}}
    \newcommand{\NormalTok}[1]{{#1}}
    
    % Additional commands for more recent versions of Pandoc
    \newcommand{\ConstantTok}[1]{\textcolor[rgb]{0.53,0.00,0.00}{{#1}}}
    \newcommand{\SpecialCharTok}[1]{\textcolor[rgb]{0.25,0.44,0.63}{{#1}}}
    \newcommand{\VerbatimStringTok}[1]{\textcolor[rgb]{0.25,0.44,0.63}{{#1}}}
    \newcommand{\SpecialStringTok}[1]{\textcolor[rgb]{0.73,0.40,0.53}{{#1}}}
    \newcommand{\ImportTok}[1]{{#1}}
    \newcommand{\DocumentationTok}[1]{\textcolor[rgb]{0.73,0.13,0.13}{\textit{{#1}}}}
    \newcommand{\AnnotationTok}[1]{\textcolor[rgb]{0.38,0.63,0.69}{\textbf{\textit{{#1}}}}}
    \newcommand{\CommentVarTok}[1]{\textcolor[rgb]{0.38,0.63,0.69}{\textbf{\textit{{#1}}}}}
    \newcommand{\VariableTok}[1]{\textcolor[rgb]{0.10,0.09,0.49}{{#1}}}
    \newcommand{\ControlFlowTok}[1]{\textcolor[rgb]{0.00,0.44,0.13}{\textbf{{#1}}}}
    \newcommand{\OperatorTok}[1]{\textcolor[rgb]{0.40,0.40,0.40}{{#1}}}
    \newcommand{\BuiltInTok}[1]{{#1}}
    \newcommand{\ExtensionTok}[1]{{#1}}
    \newcommand{\PreprocessorTok}[1]{\textcolor[rgb]{0.74,0.48,0.00}{{#1}}}
    \newcommand{\AttributeTok}[1]{\textcolor[rgb]{0.49,0.56,0.16}{{#1}}}
    \newcommand{\InformationTok}[1]{\textcolor[rgb]{0.38,0.63,0.69}{\textbf{\textit{{#1}}}}}
    \newcommand{\WarningTok}[1]{\textcolor[rgb]{0.38,0.63,0.69}{\textbf{\textit{{#1}}}}}
    
    
    % Define a nice break command that doesn't care if a line doesn't already
    % exist.
    \def\br{\hspace*{\fill} \\* }
    % Math Jax compatability definitions
    \def\gt{>}
    \def\lt{<}
    % Document parameters
    \title{06\_assignment\_odes}
    \author{Debaraj Barua}
    
    
    

    % Pygments definitions
    
\makeatletter
\def\PY@reset{\let\PY@it=\relax \let\PY@bf=\relax%
    \let\PY@ul=\relax \let\PY@tc=\relax%
    \let\PY@bc=\relax \let\PY@ff=\relax}
\def\PY@tok#1{\csname PY@tok@#1\endcsname}
\def\PY@toks#1+{\ifx\relax#1\empty\else%
    \PY@tok{#1}\expandafter\PY@toks\fi}
\def\PY@do#1{\PY@bc{\PY@tc{\PY@ul{%
    \PY@it{\PY@bf{\PY@ff{#1}}}}}}}
\def\PY#1#2{\PY@reset\PY@toks#1+\relax+\PY@do{#2}}

\expandafter\def\csname PY@tok@gd\endcsname{\def\PY@tc##1{\textcolor[rgb]{0.63,0.00,0.00}{##1}}}
\expandafter\def\csname PY@tok@gu\endcsname{\let\PY@bf=\textbf\def\PY@tc##1{\textcolor[rgb]{0.50,0.00,0.50}{##1}}}
\expandafter\def\csname PY@tok@gt\endcsname{\def\PY@tc##1{\textcolor[rgb]{0.00,0.27,0.87}{##1}}}
\expandafter\def\csname PY@tok@gs\endcsname{\let\PY@bf=\textbf}
\expandafter\def\csname PY@tok@gr\endcsname{\def\PY@tc##1{\textcolor[rgb]{1.00,0.00,0.00}{##1}}}
\expandafter\def\csname PY@tok@cm\endcsname{\let\PY@it=\textit\def\PY@tc##1{\textcolor[rgb]{0.25,0.50,0.50}{##1}}}
\expandafter\def\csname PY@tok@vg\endcsname{\def\PY@tc##1{\textcolor[rgb]{0.10,0.09,0.49}{##1}}}
\expandafter\def\csname PY@tok@vi\endcsname{\def\PY@tc##1{\textcolor[rgb]{0.10,0.09,0.49}{##1}}}
\expandafter\def\csname PY@tok@vm\endcsname{\def\PY@tc##1{\textcolor[rgb]{0.10,0.09,0.49}{##1}}}
\expandafter\def\csname PY@tok@mh\endcsname{\def\PY@tc##1{\textcolor[rgb]{0.40,0.40,0.40}{##1}}}
\expandafter\def\csname PY@tok@cs\endcsname{\let\PY@it=\textit\def\PY@tc##1{\textcolor[rgb]{0.25,0.50,0.50}{##1}}}
\expandafter\def\csname PY@tok@ge\endcsname{\let\PY@it=\textit}
\expandafter\def\csname PY@tok@vc\endcsname{\def\PY@tc##1{\textcolor[rgb]{0.10,0.09,0.49}{##1}}}
\expandafter\def\csname PY@tok@il\endcsname{\def\PY@tc##1{\textcolor[rgb]{0.40,0.40,0.40}{##1}}}
\expandafter\def\csname PY@tok@go\endcsname{\def\PY@tc##1{\textcolor[rgb]{0.53,0.53,0.53}{##1}}}
\expandafter\def\csname PY@tok@cp\endcsname{\def\PY@tc##1{\textcolor[rgb]{0.74,0.48,0.00}{##1}}}
\expandafter\def\csname PY@tok@gi\endcsname{\def\PY@tc##1{\textcolor[rgb]{0.00,0.63,0.00}{##1}}}
\expandafter\def\csname PY@tok@gh\endcsname{\let\PY@bf=\textbf\def\PY@tc##1{\textcolor[rgb]{0.00,0.00,0.50}{##1}}}
\expandafter\def\csname PY@tok@ni\endcsname{\let\PY@bf=\textbf\def\PY@tc##1{\textcolor[rgb]{0.60,0.60,0.60}{##1}}}
\expandafter\def\csname PY@tok@nl\endcsname{\def\PY@tc##1{\textcolor[rgb]{0.63,0.63,0.00}{##1}}}
\expandafter\def\csname PY@tok@nn\endcsname{\let\PY@bf=\textbf\def\PY@tc##1{\textcolor[rgb]{0.00,0.00,1.00}{##1}}}
\expandafter\def\csname PY@tok@no\endcsname{\def\PY@tc##1{\textcolor[rgb]{0.53,0.00,0.00}{##1}}}
\expandafter\def\csname PY@tok@na\endcsname{\def\PY@tc##1{\textcolor[rgb]{0.49,0.56,0.16}{##1}}}
\expandafter\def\csname PY@tok@nb\endcsname{\def\PY@tc##1{\textcolor[rgb]{0.00,0.50,0.00}{##1}}}
\expandafter\def\csname PY@tok@nc\endcsname{\let\PY@bf=\textbf\def\PY@tc##1{\textcolor[rgb]{0.00,0.00,1.00}{##1}}}
\expandafter\def\csname PY@tok@nd\endcsname{\def\PY@tc##1{\textcolor[rgb]{0.67,0.13,1.00}{##1}}}
\expandafter\def\csname PY@tok@ne\endcsname{\let\PY@bf=\textbf\def\PY@tc##1{\textcolor[rgb]{0.82,0.25,0.23}{##1}}}
\expandafter\def\csname PY@tok@nf\endcsname{\def\PY@tc##1{\textcolor[rgb]{0.00,0.00,1.00}{##1}}}
\expandafter\def\csname PY@tok@si\endcsname{\let\PY@bf=\textbf\def\PY@tc##1{\textcolor[rgb]{0.73,0.40,0.53}{##1}}}
\expandafter\def\csname PY@tok@s2\endcsname{\def\PY@tc##1{\textcolor[rgb]{0.73,0.13,0.13}{##1}}}
\expandafter\def\csname PY@tok@nt\endcsname{\let\PY@bf=\textbf\def\PY@tc##1{\textcolor[rgb]{0.00,0.50,0.00}{##1}}}
\expandafter\def\csname PY@tok@nv\endcsname{\def\PY@tc##1{\textcolor[rgb]{0.10,0.09,0.49}{##1}}}
\expandafter\def\csname PY@tok@s1\endcsname{\def\PY@tc##1{\textcolor[rgb]{0.73,0.13,0.13}{##1}}}
\expandafter\def\csname PY@tok@dl\endcsname{\def\PY@tc##1{\textcolor[rgb]{0.73,0.13,0.13}{##1}}}
\expandafter\def\csname PY@tok@ch\endcsname{\let\PY@it=\textit\def\PY@tc##1{\textcolor[rgb]{0.25,0.50,0.50}{##1}}}
\expandafter\def\csname PY@tok@m\endcsname{\def\PY@tc##1{\textcolor[rgb]{0.40,0.40,0.40}{##1}}}
\expandafter\def\csname PY@tok@gp\endcsname{\let\PY@bf=\textbf\def\PY@tc##1{\textcolor[rgb]{0.00,0.00,0.50}{##1}}}
\expandafter\def\csname PY@tok@sh\endcsname{\def\PY@tc##1{\textcolor[rgb]{0.73,0.13,0.13}{##1}}}
\expandafter\def\csname PY@tok@ow\endcsname{\let\PY@bf=\textbf\def\PY@tc##1{\textcolor[rgb]{0.67,0.13,1.00}{##1}}}
\expandafter\def\csname PY@tok@sx\endcsname{\def\PY@tc##1{\textcolor[rgb]{0.00,0.50,0.00}{##1}}}
\expandafter\def\csname PY@tok@bp\endcsname{\def\PY@tc##1{\textcolor[rgb]{0.00,0.50,0.00}{##1}}}
\expandafter\def\csname PY@tok@c1\endcsname{\let\PY@it=\textit\def\PY@tc##1{\textcolor[rgb]{0.25,0.50,0.50}{##1}}}
\expandafter\def\csname PY@tok@fm\endcsname{\def\PY@tc##1{\textcolor[rgb]{0.00,0.00,1.00}{##1}}}
\expandafter\def\csname PY@tok@o\endcsname{\def\PY@tc##1{\textcolor[rgb]{0.40,0.40,0.40}{##1}}}
\expandafter\def\csname PY@tok@kc\endcsname{\let\PY@bf=\textbf\def\PY@tc##1{\textcolor[rgb]{0.00,0.50,0.00}{##1}}}
\expandafter\def\csname PY@tok@c\endcsname{\let\PY@it=\textit\def\PY@tc##1{\textcolor[rgb]{0.25,0.50,0.50}{##1}}}
\expandafter\def\csname PY@tok@mf\endcsname{\def\PY@tc##1{\textcolor[rgb]{0.40,0.40,0.40}{##1}}}
\expandafter\def\csname PY@tok@err\endcsname{\def\PY@bc##1{\setlength{\fboxsep}{0pt}\fcolorbox[rgb]{1.00,0.00,0.00}{1,1,1}{\strut ##1}}}
\expandafter\def\csname PY@tok@mb\endcsname{\def\PY@tc##1{\textcolor[rgb]{0.40,0.40,0.40}{##1}}}
\expandafter\def\csname PY@tok@ss\endcsname{\def\PY@tc##1{\textcolor[rgb]{0.10,0.09,0.49}{##1}}}
\expandafter\def\csname PY@tok@sr\endcsname{\def\PY@tc##1{\textcolor[rgb]{0.73,0.40,0.53}{##1}}}
\expandafter\def\csname PY@tok@mo\endcsname{\def\PY@tc##1{\textcolor[rgb]{0.40,0.40,0.40}{##1}}}
\expandafter\def\csname PY@tok@kd\endcsname{\let\PY@bf=\textbf\def\PY@tc##1{\textcolor[rgb]{0.00,0.50,0.00}{##1}}}
\expandafter\def\csname PY@tok@mi\endcsname{\def\PY@tc##1{\textcolor[rgb]{0.40,0.40,0.40}{##1}}}
\expandafter\def\csname PY@tok@kn\endcsname{\let\PY@bf=\textbf\def\PY@tc##1{\textcolor[rgb]{0.00,0.50,0.00}{##1}}}
\expandafter\def\csname PY@tok@cpf\endcsname{\let\PY@it=\textit\def\PY@tc##1{\textcolor[rgb]{0.25,0.50,0.50}{##1}}}
\expandafter\def\csname PY@tok@kr\endcsname{\let\PY@bf=\textbf\def\PY@tc##1{\textcolor[rgb]{0.00,0.50,0.00}{##1}}}
\expandafter\def\csname PY@tok@s\endcsname{\def\PY@tc##1{\textcolor[rgb]{0.73,0.13,0.13}{##1}}}
\expandafter\def\csname PY@tok@kp\endcsname{\def\PY@tc##1{\textcolor[rgb]{0.00,0.50,0.00}{##1}}}
\expandafter\def\csname PY@tok@w\endcsname{\def\PY@tc##1{\textcolor[rgb]{0.73,0.73,0.73}{##1}}}
\expandafter\def\csname PY@tok@kt\endcsname{\def\PY@tc##1{\textcolor[rgb]{0.69,0.00,0.25}{##1}}}
\expandafter\def\csname PY@tok@sc\endcsname{\def\PY@tc##1{\textcolor[rgb]{0.73,0.13,0.13}{##1}}}
\expandafter\def\csname PY@tok@sb\endcsname{\def\PY@tc##1{\textcolor[rgb]{0.73,0.13,0.13}{##1}}}
\expandafter\def\csname PY@tok@sa\endcsname{\def\PY@tc##1{\textcolor[rgb]{0.73,0.13,0.13}{##1}}}
\expandafter\def\csname PY@tok@k\endcsname{\let\PY@bf=\textbf\def\PY@tc##1{\textcolor[rgb]{0.00,0.50,0.00}{##1}}}
\expandafter\def\csname PY@tok@se\endcsname{\let\PY@bf=\textbf\def\PY@tc##1{\textcolor[rgb]{0.73,0.40,0.13}{##1}}}
\expandafter\def\csname PY@tok@sd\endcsname{\let\PY@it=\textit\def\PY@tc##1{\textcolor[rgb]{0.73,0.13,0.13}{##1}}}

\def\PYZbs{\char`\\}
\def\PYZus{\char`\_}
\def\PYZob{\char`\{}
\def\PYZcb{\char`\}}
\def\PYZca{\char`\^}
\def\PYZam{\char`\&}
\def\PYZlt{\char`\<}
\def\PYZgt{\char`\>}
\def\PYZsh{\char`\#}
\def\PYZpc{\char`\%}
\def\PYZdl{\char`\$}
\def\PYZhy{\char`\-}
\def\PYZsq{\char`\'}
\def\PYZdq{\char`\"}
\def\PYZti{\char`\~}
% for compatibility with earlier versions
\def\PYZat{@}
\def\PYZlb{[}
\def\PYZrb{]}
\makeatother


    % Exact colors from NB
    \definecolor{incolor}{rgb}{0.0, 0.0, 0.5}
    \definecolor{outcolor}{rgb}{0.545, 0.0, 0.0}



    
    % Prevent overflowing lines due to hard-to-break entities
    \sloppy 
    % Setup hyperref package
    \hypersetup{
      breaklinks=true,  % so long urls are correctly broken across lines
      colorlinks=true,
      urlcolor=urlcolor,
      linkcolor=linkcolor,
      citecolor=citecolor,
      }
    % Slightly bigger margins than the latex defaults
    
    \geometry{verbose,tmargin=1in,bmargin=1in,lmargin=1in,rmargin=1in}
    
    

    \begin{document}
    
    
    \maketitle
    
    

    
    

    \section{Hochschule Bonn-Rhein-Sieg}\label{hochschule-bonn-rhein-sieg}

\section{Mathematics for Robotics and Control,
SS17}\label{mathematics-for-robotics-and-control-ss17}

\section{Assignment 6: Ordinary Differential
Equations}\label{assignment-6-ordinary-differential-equations}

    Familiarize yourself with SymPy:

\href{http://docs.sympy.org/latest/tutorial/}{SymPy Tutorial}

In particular, look at how to solve differential equations with SymPy:
\href{http://docs.sympy.org/latest/tutorial/solvers.html\#solving-differential-equations}{Differential
equations with SymPy}

    \begin{Verbatim}[commandchars=\\\{\}]
{\color{incolor}In [{\color{incolor}53}]:} \PY{k+kn}{import} \PY{n+nn}{sympy} \PY{k+kn}{as} \PY{n+nn}{sp}
         \PY{k+kn}{import} \PY{n+nn}{numpy} \PY{k+kn}{as} \PY{n+nn}{np}
         \PY{c+c1}{\PYZsh{} Enable latex printing}
         \PY{n}{sympy}\PY{o}{.}\PY{n}{init\PYZus{}printing}\PY{p}{(}\PY{n}{use\PYZus{}latex}\PY{o}{=}\PY{n+nb+bp}{True}\PY{p}{)}
\end{Verbatim}

    For your first assignment, solve the following three ODEs \emph{by hand}
and verify your results using SymPy. Please write the steps of your
manual solutions using TeX syntax in the input cell below. You can find
an introduction to mathematical expressions in LaTeX
\href{http://en.wikibooks.org/wiki/LaTeX/Mathematics}{here}. In order
for the IPython notebook to evaluate your mathematical expressions,
enclose them in \$. For instance,

\begin{verbatim}
$ \sum_{i=1}^{10} t_i $ 
\end{verbatim}

will display the following expression: \(\sum_{i=1}^{10} t_i\)

    \subsubsection{1. Solve the following ordinary differential equations by
hand and verify your results using SymPy
{[}20{]}}\label{solve-the-following-ordinary-differential-equations-by-hand-and-verify-your-results-using-sympy-20}

    \textbf{Equation 1.1}: \(y^\prime = 5 \cdot y\)

    \emph{Solve Equation 1.1 by hand and write your steps here using LaTeX}

\begin{align*}
        &\frac{dy}{dx}= 5 \cdot y \\
        &\implies \frac{dy}{y}= 5 \cdot dx \\
        &\implies \int \frac{dy}{y} = \int 5 \cdot dx \\
        & \implies \ln y = 5 \cdot x + C \\
        & \implies y = e^{5x+C} \\
        & \implies y = e^{5x}\cdot e^{C} \\
        & \implies y = C_1 \cdot e^{5x}
\end{align*}

    \begin{Verbatim}[commandchars=\\\{\}]
{\color{incolor}In [{\color{incolor}9}]:} \PY{c+c1}{\PYZsh{} Insert code to verify your solution for Equation 1.1 here and evaluate it}
        
        \PY{n}{x} \PY{o}{=} \PY{n}{sp}\PY{o}{.}\PY{n}{Symbol}\PY{p}{(}\PY{l+s+s1}{\PYZsq{}}\PY{l+s+s1}{x}\PY{l+s+s1}{\PYZsq{}}\PY{p}{)}
        \PY{n}{y}\PY{o}{=}\PY{n}{sp}\PY{o}{.}\PY{n}{Function}\PY{p}{(}\PY{l+s+s1}{\PYZsq{}}\PY{l+s+s1}{y}\PY{l+s+s1}{\PYZsq{}}\PY{p}{)}
        
        \PY{n}{y\PYZus{}prime}\PY{o}{=}\PY{n}{sp}\PY{o}{.}\PY{n}{Derivative}\PY{p}{(}\PY{n}{y}\PY{p}{(}\PY{n}{x}\PY{p}{)}\PY{p}{,}\PY{n}{x}\PY{p}{)}
        \PY{n}{eq}\PY{o}{=} \PY{n}{y\PYZus{}prime} \PY{o}{\PYZhy{}} \PY{l+m+mi}{5}\PY{o}{*}\PY{n}{y}\PY{p}{(}\PY{n}{x}\PY{p}{)}
        \PY{n}{sp}\PY{o}{.}\PY{n}{init\PYZus{}printing}\PY{p}{(}\PY{n}{use\PYZus{}latex}\PY{o}{=}\PY{n+nb+bp}{True}\PY{p}{)}
        \PY{n}{sp}\PY{o}{.}\PY{n}{dsolve}\PY{p}{(}\PY{n}{eq}\PY{p}{)}
\end{Verbatim}
\texttt{\color{outcolor}Out[{\color{outcolor}9}]:}
    
    \[y{\left (x \right )} = C_{1} e^{5 x}\]

    

    \begin{center}\rule{0.5\linewidth}{\linethickness}\end{center}

\textbf{Equation 1.2:}
\(\frac{\mathrm d y}{\mathrm d x} = -2 \cdot x \cdot y\)

    \emph{Solve Equation 1.2 by hand and write your steps here using LaTeX}

\begin{align*}
        &\frac{dy}{dx}=-2 \cdot x \cdot y \\
        &\implies \frac{dy}{y}= -2 \cdot x \cdot dx \\
        &\implies \int \frac{dy}{y} = \int -2 \cdot x \cdot dx \\
        & \implies \ln y = -2 \cdot \frac{x^2}{2} + C \\
        & \implies y = e^{-{x^2}+C} \\
        & \implies y = e^{-{x^2}}\cdot e^{C} \\
        & \implies y = C_1 \cdot e^{-{x^2}}
\end{align*}

    \begin{Verbatim}[commandchars=\\\{\}]
{\color{incolor}In [{\color{incolor}10}]:} \PY{c+c1}{\PYZsh{} Insert code to verify your solution for Equation 1.2 here and evaluate it}
         
         \PY{n}{x} \PY{o}{=} \PY{n}{sp}\PY{o}{.}\PY{n}{Symbol}\PY{p}{(}\PY{l+s+s1}{\PYZsq{}}\PY{l+s+s1}{x}\PY{l+s+s1}{\PYZsq{}}\PY{p}{)}
         \PY{n}{y}\PY{o}{=}\PY{n}{sp}\PY{o}{.}\PY{n}{Function}\PY{p}{(}\PY{l+s+s1}{\PYZsq{}}\PY{l+s+s1}{y}\PY{l+s+s1}{\PYZsq{}}\PY{p}{)}
         
         \PY{n}{y\PYZus{}prime}\PY{o}{=}\PY{n}{sp}\PY{o}{.}\PY{n}{Derivative}\PY{p}{(}\PY{n}{y}\PY{p}{(}\PY{n}{x}\PY{p}{)}\PY{p}{,}\PY{n}{x}\PY{p}{)}
         \PY{n}{eq}\PY{o}{=}\PY{n}{y\PYZus{}prime}\PY{o}{+}\PY{l+m+mi}{2}\PY{o}{*}\PY{n}{x}\PY{o}{*}\PY{n}{y}\PY{p}{(}\PY{n}{x}\PY{p}{)}
         \PY{n}{sp}\PY{o}{.}\PY{n}{init\PYZus{}printing}\PY{p}{(}\PY{n}{use\PYZus{}latex}\PY{o}{=}\PY{n+nb+bp}{True}\PY{p}{)}
         \PY{n}{sp}\PY{o}{.}\PY{n}{dsolve}\PY{p}{(}\PY{n}{eq}\PY{p}{)}
\end{Verbatim}
\texttt{\color{outcolor}Out[{\color{outcolor}10}]:}
    
    \[y{\left (x \right )} = C_{1} e^{- x^{2}}\]

    

    \begin{center}\rule{0.5\linewidth}{\linethickness}\end{center}

\textbf{Equation 1.3:} \(\frac{\mathrm d y}{\mathrm d x} = y^2\)

    \emph{Solve Equation 1.3 by hand and write your steps here using LaTeX}

\begin{align*}
        &\frac{dy}{dx}= y^2 \\
        &\implies \frac{dy}{y^2}= dx \\
        &\implies \int \frac{dy}{y^2} = \int dx \\
        & \implies -\frac{1}{y} = x + C_1 \\
        & \implies y = -\frac{1}{x+C_1} 
\end{align*}

    \begin{Verbatim}[commandchars=\\\{\}]
{\color{incolor}In [{\color{incolor}11}]:} \PY{c+c1}{\PYZsh{} Insert code to verify your solution for Equation 1.3 here and evaluate it}
         
         \PY{n}{x} \PY{o}{=} \PY{n}{sp}\PY{o}{.}\PY{n}{Symbol}\PY{p}{(}\PY{l+s+s1}{\PYZsq{}}\PY{l+s+s1}{x}\PY{l+s+s1}{\PYZsq{}}\PY{p}{)}
         \PY{n}{y}\PY{o}{=}\PY{n}{sp}\PY{o}{.}\PY{n}{Function}\PY{p}{(}\PY{l+s+s1}{\PYZsq{}}\PY{l+s+s1}{y}\PY{l+s+s1}{\PYZsq{}}\PY{p}{)}
         
         \PY{n}{y\PYZus{}prime}\PY{o}{=}\PY{n}{sp}\PY{o}{.}\PY{n}{Derivative}\PY{p}{(}\PY{n}{y}\PY{p}{(}\PY{n}{x}\PY{p}{)}\PY{p}{,}\PY{n}{x}\PY{p}{)}
         \PY{n}{eq}\PY{o}{=}\PY{n}{y\PYZus{}prime}\PY{o}{\PYZhy{}}\PY{n}{y}\PY{p}{(}\PY{n}{x}\PY{p}{)}\PY{o}{*}\PY{o}{*}\PY{l+m+mi}{2}
         \PY{n}{sp}\PY{o}{.}\PY{n}{init\PYZus{}printing}\PY{p}{(}\PY{n}{use\PYZus{}latex}\PY{o}{=}\PY{n+nb+bp}{True}\PY{p}{)}
         \PY{n}{sp}\PY{o}{.}\PY{n}{dsolve}\PY{p}{(}\PY{n}{eq}\PY{p}{)}
\end{Verbatim}
\texttt{\color{outcolor}Out[{\color{outcolor}11}]:}
    
    \[y{\left (x \right )} = - \frac{1}{C_{1} + x}\]

    

    \begin{center}\rule{0.5\linewidth}{\linethickness}\end{center}

\textbf{Equation 1.4}: \(y^\prime + \frac{4}{x} \cdot y = x^4\)

    \emph{Solve Equation 1.4 by hand and write your steps here using LaTeX}

We have, \[
\frac{dy}{dx} + \frac{4}{x} \cdot y = x^4
\]

Now, let us assume, a function \(f(x)\) such that $\frac{df}{dx} = f(x)
\cdot  \frac{4}{x} $. So,

\begin{align*}
        & \frac {df}{dx} =  f \cdot  \frac{4}{x}\\
        & \implies \frac {df}{f} =  \frac{4}{x} \cdot dx\\
        & \implies \int \frac {df}{f} =  \int \frac{4}{x} \cdot dx\\
        & \implies \ln{f} = 4 \cdot \ln{x} + C_1\\
        & \implies f = e^{4 \cdot \ln{x} + C_1}\\
        & \implies f = {(e^{\ln{x}})}^4 \cdot e^{C_1}\\
        & \implies f = x^4 \cdot C_2\\
\end{align*}

Therefore our equation can be re-written as:

\begin{align*}
        & f \cdot \frac{dy}{dx} + f \cdot \frac{4}{x} \cdot y =f \cdot x^4\\
        & \implies f \cdot \frac{dy}{dx} + \frac{df}{dx} \cdot y =f \cdot x^4\\
        & \implies \frac{d}{dx}(f \cdot y) = f \cdot x^4 \\
        & \implies \int {d}(f \cdot y) =  \int f \cdot x^4 {dx} \\
        & \implies f \cdot y =  \int f \cdot x^4 {dx} \\
        & \implies y =  \frac{\int f \cdot x^4 {dx}}{f} \\
\end{align*}

Using the equation \(f(x)= x^4 \cdot C_2\),

\begin{align*}
        & y =  \frac{\int f \cdot x^4 {dx}}{f} \\
        & \implies y = \frac{\int  x^8 \cdot C_2 \cdot {dx}}{ x^4 \cdot C_2} \\
        & \implies y = \frac{ \frac {x^9}{9} + C_3}{ x^4} \\
        & \implies y = \frac{C_3}{ x^4} + \frac {x^5}{9}  \\
\end{align*}

    \begin{Verbatim}[commandchars=\\\{\}]
{\color{incolor}In [{\color{incolor}8}]:} \PY{c+c1}{\PYZsh{} Insert code to verify your solution for Equation 1.4 here and evaluate it}
        
        
        \PY{n}{x} \PY{o}{=} \PY{n}{sp}\PY{o}{.}\PY{n}{Symbol}\PY{p}{(}\PY{l+s+s1}{\PYZsq{}}\PY{l+s+s1}{x}\PY{l+s+s1}{\PYZsq{}}\PY{p}{)}
        \PY{n}{y}\PY{o}{=}\PY{n}{sp}\PY{o}{.}\PY{n}{Function}\PY{p}{(}\PY{l+s+s1}{\PYZsq{}}\PY{l+s+s1}{y}\PY{l+s+s1}{\PYZsq{}}\PY{p}{)}
        
        \PY{n}{y\PYZus{}prime}\PY{o}{=}\PY{n}{sp}\PY{o}{.}\PY{n}{Derivative}\PY{p}{(}\PY{n}{y}\PY{p}{(}\PY{n}{x}\PY{p}{)}\PY{p}{,}\PY{n}{x}\PY{p}{)}
        \PY{n}{eq}\PY{o}{=}\PY{n}{y\PYZus{}prime}\PY{o}{+}\PY{l+m+mi}{4}\PY{o}{/}\PY{n}{x}\PY{o}{*}\PY{n}{y}\PY{p}{(}\PY{n}{x}\PY{p}{)}\PY{o}{\PYZhy{}}\PY{n}{x}\PY{o}{*}\PY{o}{*}\PY{l+m+mi}{4}
        \PY{n}{sp}\PY{o}{.}\PY{n}{init\PYZus{}printing}\PY{p}{(}\PY{n}{use\PYZus{}latex}\PY{o}{=}\PY{n+nb+bp}{True}\PY{p}{)}
        \PY{n}{sp}\PY{o}{.}\PY{n}{dsolve}\PY{p}{(}\PY{n}{eq}\PY{p}{)}
\end{Verbatim}
\texttt{\color{outcolor}Out[{\color{outcolor}8}]:}
    
    \[y{\left (x \right )} = \frac{C_{1}}{x^{4}} + \frac{x^{5}}{9}\]

    

    \begin{center}\rule{0.5\linewidth}{\linethickness}\end{center}

\begin{center}\rule{0.5\linewidth}{\linethickness}\end{center}

\subsubsection{Numerical methods for solving differential equations
{[}50{]}}\label{numerical-methods-for-solving-differential-equations-50}

In the lab class, you used Euler's method to numerically estimate the
solution to an ODE; this method belongs to the family of Runge-Kutta
methods. Here you will use the fourth-order Runge-Kutta method to solve
an ODE numerically. The higher order Runge-Kutta methods are more
accurate since they also consider the slopes at points between the
current and next step (instead of just the slope at the current point).

The relevant equations needed are: \[ y^\prime = f(x, y) \]
\[ x_{n+1} = x_n + h \]
\[ y_{n+1} = y_n + \frac{1}{6}[k_1 + 2k_2 + 2k_3 + k_4]\]
\[k_1 = h \cdot f(x_n,y_n)\]
\[k_2 = h \cdot f(x_n + \frac{h}{2}, y + \frac{k_1}{2})\]
\[k_3 = h \cdot f(x_n + \frac{h}{2}, y + \frac{k_2}{2})\]
\[k_4 = h \cdot f(x_n + h, y + k_3)\]

The context of the exercise is as follows: a robot is tasked with
changing a lamp in a factory. The task involves pressing a switch to
turn the lamp off, waiting until it cools down to \(30^{\circ}\) C, and
then unscrewing it. The robot is able to query the temperature of the
lamp as long as it is turned on, but once the lamp is turned off, the
temperature cannot be queried.

Newton's law of cooling states that the rate of change of temperature of
an object is directly proportional to the difference between the
temperature of the object and the ambient temperature. This can be
written in the form of a differential equation as follows:

\[\frac{dT}{dt} = -K(T - T_A)\] where \(T\) is the temperature of the
object, \(T_A\) is the ambient temperature and \(K\) is a constant that
depends on the properties of the object.

The robot queries the temperature of the lamp just before turning it off
and receives \(T_{t=0} = 150^{\circ}\) C. The ambient temperature is
known to be \(T_A = 20^{\circ}\) C. The robot also knows that for this
lamp, \(K = 0.05\).

The robot needs to calculate how long it needs to wait until the lamp
has cooled down sufficiently; i.e. find the \(t\) when the temperature
is just below \(30^{\circ}\) C.

\textbf{Solve this problem using fourth-order Runge-Kutta, using step
sizes 1, 0.1 and 0.01.}

    \begin{Verbatim}[commandchars=\\\{\}]
{\color{incolor}In [{\color{incolor}54}]:} \PY{c+c1}{\PYZsh{} Implement the Runge\PYZhy{}Kutta method here to solve }
         \PY{c+c1}{\PYZsh{} the given differential equation}
         
         \PY{n}{K} \PY{o}{=} \PY{l+m+mf}{0.05}
         \PY{c+c1}{\PYZsh{}h = 0.01 \PYZsh{} also show the result with h = 1 and 0.1}
         \PY{n}{T\PYZus{}A} \PY{o}{=} \PY{l+m+mf}{20.0}
         \PY{n}{stop\PYZus{}temp} \PY{o}{=} \PY{l+m+mf}{30.0}
         \PY{n}{start\PYZus{}temp} \PY{o}{=} \PY{l+m+mf}{150.0}  
         
         \PY{l+s+sd}{\PYZdq{}\PYZdq{}\PYZdq{}}
         \PY{l+s+sd}{To find t when T=30}
         \PY{l+s+sd}{\PYZdq{}\PYZdq{}\PYZdq{}}
         
         \PY{l+s+sd}{\PYZdq{}\PYZdq{}\PYZdq{}}
         \PY{l+s+sd}{T\PYZus{}prime: Function gives the value of the derivative dT/dt }
         \PY{l+s+sd}{         at Temperature \PYZdq{}Temp\PYZdq{} and time \PYZdq{}t\PYZdq{}}
         \PY{l+s+sd}{\PYZdq{}\PYZdq{}\PYZdq{}}
         \PY{k}{def} \PY{n+nf}{T\PYZus{}prime}\PY{p}{(}\PY{n}{t}\PY{p}{,} \PY{n}{Temp}\PY{p}{)}\PY{p}{:}
             \PY{k}{return} \PY{o}{\PYZhy{}}\PY{n}{K}\PY{o}{*}\PY{p}{(}\PY{n}{Temp}\PY{o}{\PYZhy{}}\PY{n}{T\PYZus{}A}\PY{p}{)}
         
         
         \PY{l+s+sd}{\PYZdq{}\PYZdq{}\PYZdq{}}
         \PY{l+s+sd}{R\PYZus{}K : Function returns the time estimate using fourth\PYZhy{}order Runge\PYZhy{}Kutta }
         \PY{l+s+sd}{ic  : Initial values ic[0]: time at t\PYZus{}0, and ic[1]: Temperature at t\PYZus{}0}
         \PY{l+s+sd}{h   : Step size}
         \PY{l+s+sd}{\PYZdq{}\PYZdq{}\PYZdq{}}
         \PY{k}{def} \PY{n+nf}{R\PYZus{}K}\PY{p}{(}\PY{n}{ic}\PY{p}{,} \PY{n}{h}\PY{p}{)}\PY{p}{:}
             
             \PY{n}{t\PYZus{}current} \PY{o}{=} \PY{n}{ic}\PY{p}{[}\PY{l+m+mi}{0}\PY{p}{]}
             \PY{n}{Temp} \PY{o}{=} \PY{n}{ic}\PY{p}{[}\PY{l+m+mi}{1}\PY{p}{]}
            
             \PY{c+c1}{\PYZsh{}Loop until current temperature is \PYZlt{}=30 degree Celsius}
             \PY{k}{while}\PY{p}{(}\PY{n}{Temp}\PY{o}{\PYZgt{}}\PY{n}{stop\PYZus{}temp}\PY{p}{)}\PY{p}{:}
                 
                 \PY{n}{k1} \PY{o}{=} \PY{n}{h}\PY{o}{*}\PY{n}{T\PYZus{}prime}\PY{p}{(}\PY{n}{t\PYZus{}current}\PY{p}{,}\PY{n}{Temp}\PY{p}{)}
                 \PY{n}{k2} \PY{o}{=} \PY{n}{h}\PY{o}{*}\PY{n}{T\PYZus{}prime}\PY{p}{(}\PY{n}{t\PYZus{}current}\PY{o}{+}\PY{n}{h}\PY{o}{/}\PY{l+m+mi}{2}\PY{p}{,}\PY{n}{Temp}\PY{o}{+}\PY{l+m+mi}{1}\PY{o}{/}\PY{l+m+mi}{2}\PY{o}{*}\PY{n}{k1}\PY{p}{)}
                 \PY{n}{k3} \PY{o}{=} \PY{n}{h}\PY{o}{*}\PY{n}{T\PYZus{}prime}\PY{p}{(}\PY{n}{t\PYZus{}current}\PY{o}{+}\PY{n}{h}\PY{o}{/}\PY{l+m+mi}{2}\PY{p}{,}\PY{n}{Temp}\PY{o}{+}\PY{l+m+mi}{1}\PY{o}{/}\PY{l+m+mi}{2}\PY{o}{*}\PY{n}{k2}\PY{p}{)}
                 \PY{n}{k4} \PY{o}{=} \PY{n}{h}\PY{o}{*}\PY{n}{T\PYZus{}prime}\PY{p}{(}\PY{n}{t\PYZus{}current}\PY{o}{+}\PY{n}{h}\PY{p}{,}\PY{n}{Temp}\PY{o}{+}\PY{l+m+mi}{1}\PY{o}{*}\PY{n}{k3}\PY{p}{)}
                 
                 \PY{n}{k} \PY{o}{=}\PY{p}{(}\PY{n}{k1}\PY{o}{+}\PY{p}{(}\PY{l+m+mi}{2}\PY{o}{*}\PY{n}{k2}\PY{p}{)}\PY{o}{+}\PY{p}{(}\PY{l+m+mi}{2}\PY{o}{*}\PY{n}{k3}\PY{p}{)}\PY{o}{+}\PY{n}{k4}\PY{p}{)}
                 
                 \PY{n}{Temp} \PY{o}{=} \PY{p}{(}\PY{n}{Temp} \PY{o}{+} \PY{p}{(}\PY{n}{k}\PY{o}{/}\PY{l+m+mi}{6}\PY{p}{)}\PY{p}{)}   \PY{c+c1}{\PYZsh{} update temperature using RK\PYZhy{}4}
                 \PY{n}{t\PYZus{}current} \PY{o}{+}\PY{o}{=} \PY{n}{h}          \PY{c+c1}{\PYZsh{} increment time by step size}
             
             \PY{c+c1}{\PYZsh{}Return time at which temperature dropped to 30 degree Celsius}
             \PY{k}{return} \PY{n}{t\PYZus{}current}
         
         
         
         \PY{c+c1}{\PYZsh{}STEP SIZE: 1}
         \PY{n}{ic} \PY{o}{=} \PY{n}{np}\PY{o}{.}\PY{n}{array}\PY{p}{(}\PY{p}{[}\PY{l+m+mf}{0.}\PY{p}{,} \PY{n}{start\PYZus{}temp}\PY{p}{]}\PY{p}{)}
         \PY{n}{h} \PY{o}{=} \PY{l+m+mf}{1.0}
         \PY{n}{t\PYZus{}stp1} \PY{o}{=} \PY{n}{R\PYZus{}K}\PY{p}{(}\PY{n}{ic}\PY{p}{,} \PY{n}{h}\PY{p}{)}
         
         \PY{k}{print} \PY{l+s+s2}{\PYZdq{}}\PY{l+s+s2}{Time approximation with step size }\PY{l+s+s2}{\PYZdq{}}\PY{p}{,} \PY{n}{h}\PY{p}{,} \PY{l+s+s2}{\PYZdq{}}\PY{l+s+s2}{ is: }\PY{l+s+s2}{\PYZdq{}}\PY{p}{,} \PY{n}{t\PYZus{}stp1}\PY{p}{,} \PY{l+s+s2}{\PYZdq{}}\PY{l+s+s2}{seconds }\PY{l+s+s2}{\PYZdq{}}
         
         \PY{c+c1}{\PYZsh{}STEP SIZE: 0.1}
         \PY{n}{ic} \PY{o}{=} \PY{n}{np}\PY{o}{.}\PY{n}{array}\PY{p}{(}\PY{p}{[}\PY{l+m+mf}{0.}\PY{p}{,} \PY{n}{start\PYZus{}temp}\PY{p}{]}\PY{p}{)}
         \PY{n}{h} \PY{o}{=} \PY{l+m+mf}{0.1}
         \PY{n}{t\PYZus{}stp2} \PY{o}{=} \PY{n}{R\PYZus{}K}\PY{p}{(}\PY{n}{ic}\PY{p}{,} \PY{n}{h}\PY{p}{)}
         
         \PY{k}{print} \PY{l+s+s2}{\PYZdq{}}\PY{l+s+s2}{Time approximation with step size }\PY{l+s+s2}{\PYZdq{}}\PY{p}{,} \PY{n}{h}\PY{p}{,} \PY{l+s+s2}{\PYZdq{}}\PY{l+s+s2}{ is: }\PY{l+s+s2}{\PYZdq{}}\PY{p}{,} \PY{n}{t\PYZus{}stp2}\PY{p}{,} \PY{l+s+s2}{\PYZdq{}}\PY{l+s+s2}{seconds }\PY{l+s+s2}{\PYZdq{}}
         
         \PY{c+c1}{\PYZsh{}STEP SIZE: 0.01}
         \PY{n}{ic} \PY{o}{=} \PY{n}{np}\PY{o}{.}\PY{n}{array}\PY{p}{(}\PY{p}{[}\PY{l+m+mf}{0.}\PY{p}{,} \PY{n}{start\PYZus{}temp}\PY{p}{]}\PY{p}{)}
         \PY{n}{h} \PY{o}{=} \PY{l+m+mf}{0.01}
         \PY{n}{t\PYZus{}stp3} \PY{o}{=} \PY{n}{R\PYZus{}K}\PY{p}{(}\PY{n}{ic}\PY{p}{,} \PY{n}{h}\PY{p}{)}
         
         \PY{k}{print} \PY{l+s+s2}{\PYZdq{}}\PY{l+s+s2}{Time approximation with step size }\PY{l+s+s2}{\PYZdq{}}\PY{p}{,} \PY{n}{h}\PY{p}{,} \PY{l+s+s2}{\PYZdq{}}\PY{l+s+s2}{ is: }\PY{l+s+s2}{\PYZdq{}}\PY{p}{,} \PY{n}{t\PYZus{}stp3}\PY{p}{,} \PY{l+s+s2}{\PYZdq{}}\PY{l+s+s2}{seconds }\PY{l+s+s2}{\PYZdq{}}
\end{Verbatim}

    \begin{Verbatim}[commandchars=\\\{\}]
Time approximation with step size  1.0  is:  51.0 seconds 
Time approximation with step size  0.1  is:  51.3 seconds 
Time approximation with step size  0.01  is:  51.3 seconds 

    \end{Verbatim}

    \subsubsection{Verification using SymPy and application of initial
conditions
{[}10{]}}\label{verification-using-sympy-and-application-of-initial-conditions-10}

Use SymPy to find the general solution to the above problem and find
\(C_1\) using the initial conditions given. Using this particular
solution, \textbf{show that the solution obtained using Runge-Kutta is
correct.}

    \begin{Verbatim}[commandchars=\\\{\}]
{\color{incolor}In [{\color{incolor}16}]:} \PY{c+c1}{\PYZsh{} Code in SymPy here}
         \PY{n}{t} \PY{o}{=} \PY{n}{sp}\PY{o}{.}\PY{n}{Symbol}\PY{p}{(}\PY{l+s+s1}{\PYZsq{}}\PY{l+s+s1}{t}\PY{l+s+s1}{\PYZsq{}}\PY{p}{)}
         \PY{n}{T}\PY{o}{=}\PY{n}{sp}\PY{o}{.}\PY{n}{Function}\PY{p}{(}\PY{l+s+s1}{\PYZsq{}}\PY{l+s+s1}{T}\PY{l+s+s1}{\PYZsq{}}\PY{p}{)}
         
         \PY{n}{T\PYZus{}prime}\PY{o}{=}\PY{n}{sp}\PY{o}{.}\PY{n}{Derivative}\PY{p}{(}\PY{n}{T}\PY{p}{(}\PY{n}{t}\PY{p}{)}\PY{p}{,}\PY{n}{t}\PY{p}{)}
         \PY{n}{eq}\PY{o}{=}\PY{n}{T\PYZus{}prime}\PY{o}{+}\PY{l+m+mf}{0.05}\PY{o}{*}\PY{p}{(}\PY{n}{T}\PY{p}{(}\PY{n}{t}\PY{p}{)}\PY{o}{\PYZhy{}}\PY{l+m+mi}{20}\PY{p}{)}
         \PY{n}{sp}\PY{o}{.}\PY{n}{init\PYZus{}printing}\PY{p}{(}\PY{n}{use\PYZus{}latex}\PY{o}{=}\PY{n+nb+bp}{True}\PY{p}{)}
         \PY{n}{sp}\PY{o}{.}\PY{n}{dsolve}\PY{p}{(}\PY{n}{eq}\PY{p}{)}
\end{Verbatim}
\texttt{\color{outcolor}Out[{\color{outcolor}16}]:}
    
    \[T{\left (t \right )} = C_{1} e^{- 0.05 t} + 20.0\]

    

    \emph{Apply initial conditions and required temperature to solve for
\(t\) here:}

Now, we have, \(T(0) = 150^{\circ}C\)

So,

\begin{align*}
    &T{(t)} = C_{1} e^{- 0.05 t} + 20.0\\
    &\implies T(0) =  C_{1} e^{- 0.05 * 0} + 20.0 \\
    &\implies 150.0 = C_1 + 20.0 \\
    &\implies C_1 = 130.0 
\end{align*}

Let \(t_i\) be the time when Temperature drops to \(30^{\circ}C\),

So,

\begin{align*}
    & 30.0 = 130.0 \cdot e^{- 0.05 t_i} + 20.0\\
    & \implies  e^{- 0.05 t_i} = \frac{10}{130} \\
    & \implies -0.05 t = \ln(\frac{1}{13}) \\
    & \implies t_i = 51.23 
\end{align*}


    % Add a bibliography block to the postdoc
    
    
    
    \end{document}
